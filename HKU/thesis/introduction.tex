\chapter{Introduction}
\label{sec:introduction}

Repetition is an integral part of nature as manifested in common phenomena such as surface patterns (e.g. walls, fabrics, floors), geometry structures (e.g. stacks, pebbles, branches), dynamic motions (e.g. fluid turbulence, walking cycles, crowd movement) and human activities (e.g. drawing, modelling, gesturing).
Repetition has been an important subject of study for many engineering and scientific disciplines, due to its ubiquity.
The main challenge is to design methods that are general and effective, and interfaces that are simple and easy to use for diverse phenomena and application domains.
Data-driven and procedural computations have been shown as promising methods, especially for digital content creation in graphics and interaction.
This thesis demonstrates three novel interactive systems for analyzing and synthesizing sketch and animation repetitions, which are all published at top venues in graphics and HCI.


