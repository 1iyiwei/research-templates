% !TEX encoding = UTF-8 Unicode
% ----------------------------------------------------------------------
%                   LATEX TEMPLATE FOR PhD THESIS
% ----------------------------------------------------------------------

% based on Harish Bhanderi's PhD/MPhil template, then Uni Cambridge
% http://www-h.eng.cam.ac.uk/help/tpl/textprocessing/ThesisStyle/
% corrected and extended in 2007 by Jakob Suckale, then MPI-CBG PhD programme
% and made available through OpenWetWare.org - the free biology wiki


%: Style file for Latex
% Most style definitions are in the external file PhDthesisPSnPDF.
% In this template package, it can be found in ./Latex/Classes/
\documentclass[twoside,11pt]{Latex/Classes/PhDthesisPSnPDF}

%: Macro file for Latex
% Macros help you summarise frequently repeated Latex commands.
% Here, they are placed in an external file /Latex/Macros/MacroFile1.tex
% An macro that you may use frequently is the figuremacro (see introduction.tex)
\include{Latex/Macros/MacroFile1}

%\documentclass{acmsiggraph}

\NeedsTeXFormat{LaTeX2e}
%\usepackage[draft]{hyperref} % new acmsiggraph.cls
\usepackage{ifthen}
\usepackage[export]{adjustbox}
\usepackage{alltt}
\usepackage{mathenv}
\usepackage{amsmath}
\usepackage{amssymb}
\usepackage{color}
\ifdefined\sigchi
\else
\usepackage{amsthm}
\fi
%\usepackage{rotating}
\usepackage{newlfont} % for Box
%\usepackage{ulem}
\usepackage{floatflt}
\usepackage{wrapfig}
\usepackage{float}
\usepackage{fixltx2e}
%\usepackage{subcaption} % \liyi{Don't use subfigure; it is obsolete; use subfloat instead.}
\usepackage{subfig} % for subfloat
\usepackage{multirow}
\usepackage{CJKutf8} % Chinese
\usepackage{booktabs}
%\usepackage{media9} % embed video

%\DeclareCaptionType{copyrightbox} % to fix the "Package caption Error: No float type 'copyrightbox' defined."
\ifdefined\uist
\usepackage[pdftex]{graphicx}
\usepackage{parskip}
\usepackage[draft]{hyperref}
\usepackage{uist}
\fi

\ifdefined\sigchi
\usepackage{parskip}
\usepackage{url}      % llt: nicely formatted URLs
\input{chi_macros}
\else
\fi

\usepackage{cleveref}


\input{macros}
%%%% institution
\newcommand{\hku}{Univ. Hong Kong}

%%%% texture
\newcommand{\textureenergyf}[0]{E}
\newcommand{\otherenergyf}[0]{\Theta}
\newcommand{\inputsample}{\sample_{i}}
\newcommand{\outputsample}{\sample_{o}}

%\newcommand{\timesition}{\mathbf{t}} %time

\newcommand{\payload}{\mathbf{q}}

\newcommand{\outputSet}[0]{\mathcal{O}}
\newcommand{\outputSubset}[0]{\outputSet{}^\dagger}
\newcommand{\inputSet}[0]{\mathcal{I}}
\newcommand{\outputSym}[0]{\mathcal{O}}
\newcommand{\inputSym}[0]{\mathcal{I}}
\renewcommand{\outputSubset}[0]{\outputSym{}}

\newcommand{\controlSym}[0]{\mathcal{C}}
\newcommand{\inputControlSym}[0]{\mathcal{C}^i}
\newcommand{\outputControlSym}[0]{\mathcal{C}^o}

% DET stuff
\newcommand{\element}{\sample}
\newcommand{\inputelement}{\element_{i}}
\newcommand{\outputelement}{\element_{o}}
\newcommand{\splatkernel}{\mathrm{\kappa}}
\newcommand{\filterkernel}{\splatkernel}


%sample representation
\newcommand{\sample}{s}
\newcommand{\op}{op}
\newcommand{\cop}{cop}
\newcommand{\point}{p}
\newcommand{\position}{\mathbf{p}}
\newcommand{\shape}{\mathbf{p}}
\newcommand{\orientation}{\mathbf{o}}
\newcommand{\motion}{\mathbf{m}}
\newcommand{\appearance}{\mathbf{a}}
\newcommand{\workflowtime}{\mathbf{t}}
\newcommand{\velocity}{\mathbf{v}}
\newcommand{\timesition}{t} %Do not use mathbf because time is not a vector.
\newcommand{\diff}[1]{\widehat{#1}}
\newcommand{\diffshape}{\diff{\shape}}
\newcommand{\diffmotion}{\diff{\motion}}
\newcommand{\diffappearance}{\diff{\appearance}}
\newcommand{\diffworkflowtime}{\diff{\workflowtime}}
\newcommand{\positiontime}{\mathbf{u}}
\newcommand{\diffpositiontime}{\diff{\positiontime}}


%energy brush
\newcommand{\constraints}{\mathbf{c}}
\newcommand{\flowparticle}{q}
\newcommand{\velocityfield}{\mathbf{F}}
\newcommand{\velocityfieldnorm}{\mathbf{f}}
\newcommand{\sizecontrol}{r}
\newcommand{\strengthcontrol}{e}
\newcommand{\previous}[1]{\primeone{#1}}

\newcommand{\globalid}{\mathbf{gt}}
\newcommand{\localid}{\mathbf{lt}}

\newcommand{\colorweight}{\alpha}
\newcommand{\timeweight}{\beta}
\newcommand{\motionweight}{\gamma}

\newcommand{\varsample}{\varsigma} % alternative symbol for $\sample$ to avoid ambiguity
\newcommand{\unionop}{\bigcup}

%neighborhood
\newcommand{\neighborhood}{\mathbf{n}}
\newcommand{\diffneighborhood}{\widehat{\neighborhood}}
\newcommand{\neighborhoodinter}{\mathbf{n_b}}
\newcommand{\neighborhoodintra}{\mathbf{n_a}}
\newcommand{\neighborhoodcontext}{\mathbf{n_c}}

\newcommand{\neighborhoodsize}{r}
\newcommand{\neighborhoodparameter}{\mathbf{c}}


\newcommand{\sampledis}{\mathbf{d}}
\newcommand{\sampledisparam}{\sigma}
\newcommand{\sampledevparam}{\sigma_{1}}

\newcommand{\contextparameter}{\mathbf{c}}
\newcommand{\contextstructure}{\mathbf{cs}}
\newcommand{\contextposition}{\mathbf{cp}}
\newcommand{\contextdirection}{\mathbf{cd}}
\newcommand{\contextappearance}{\mathbf{ca}}
\newcommand{\contextneighborhood}{\mathbf{cn}}
\newcommand{\contextmode}{\mathbf{cm}}

\newcommand{\predictionquality}{\mathbf{Q}}

\newcommand{\hintmode}{{\em hint}}
\newcommand{\automode}{{\em auto}}
\newcommand{\nobrushmode}{{\em no brush}}
\newcommand{\ctrlmode}{{\em clone}}
\newcommand{\fullmode}{{\em full}} % full manual
\newcommand{\ourmode}{{\em our}} % hint + auto


\newcommand{\probability}{\Pi}
\newcommand{\localsimilarity}{\pi}
\newcommand{\similarity}{\sigma}

%%%deformation
\newcommand{\deform}[1]{\widetilde{#1}}

\newcommand{\rotation}{\mathbf{A}}
\newcommand{\rotationelement}{a}
\newcommand{\translation}{\mathbf{\delta}} % \liyi{t is already used for time}
\newcommand{\rotatecenter}{\mathbf{g}}

\newcommand{\deformationenergy}[0]{E}
\newcommand{\deformationenergyrigid}{\deformationenergy_{r}}
\newcommand{\deformationenergysmooth}{\deformationenergy_{s}}
\newcommand{\deformationenergymotion}{\deformationenergy_{m}}
\newcommand{\deformationenergyfit}{\deformationenergy_{f}}

\newcommand{\deformationweight}{\kappa} % \liyi{$\alpha$ is already used!}
\newcommand{\deformationweightrigid}{\deformationweight_{r}}
\newcommand{\deformationweightsmooth}{\deformationweight_{s}}
\newcommand{\deformationweightmotion}{\deformationweight_{m}}
\newcommand{\deformationweightfit}{\deformationweight_{f}}



%\newcommand{\rotationpenalty}{\mathbf{Rot}} %\liyi{don't use such a long and bold face symbol for a simple scalar quantity}
\newcommand{\rotationpenalty}{\gamma}

\newcommand{\primeone}[1]{#1'} %\liyi{{\em ,} is a punctuation so do not use it for math symbol}
\newcommand{\primetwo}[1]{#1''}

\DeclareMathOperator*{\argmin}{arg\,min}
\DeclareMathOperator*{\argmax}{arg\,max}
%\newcommand{\expfunc}[1]{\mathbf{e}^{#1}}
\newcommand{\expfunc}[1]{\exp\left(#1\right)}
\newcommand{\std}{\mathrm{std}}

\newcommand{\frameconsistence}{\theta}
\newcommand{\deformationsmoothconsistence}{\theta_{1}}
\newcommand{\deformationtranslationconsistence}{\theta_{2}}
\newcommand{\deformationrotationconsistence}{\theta_{3}}
\newcommand{\deformationrotationconsistenceweight}{\varphi_{1}}
\newcommand{\correspondencevariationweight}{\theta_{4}}
\newcommand{\correspondencebalanceweight}{\varphi_{2}}


\newcommand{\currentframe}[1]{{#1}}
\newcommand{\previousframe}[1]{\primeone{#1}}
\newcommand{\prepreviousframe}[1]{\primetwo{#1}}

\newcommand{\matchinggraph}{\Phi}

\graphicspath{
{figs/logo/}
}

%: ----------------------------------------------------------------------
%:                  TITLE PAGE: name, degree,..
% ----------------------------------------------------------------------
% below is to generate the title page with crest and author name

%if output to PDF then put the following in PDF header
\ifpdf  
    \pdfinfo { /Title  (PhD and MPhil Thesis Classes)
               /Creator (TeX)
               /Producer (pdfTeX)
               /Author (Jun Xing junxnui@gmail.com)
               /CreationDate (D:20160801)  %format D:YYYYMMDDhhmmss
               /ModDate (D:20160801)
               /Subject (xyz)
               /Keywords (add, your, keywords, here) }
    \pdfcatalog { /PageMode (/UseOutlines)
                  /OpenAction (fitbh)  }
\fi

\newcommand{\zh}[1]{\begin{CJK}{UTF8}{gbsn}#1\end{CJK}}
\newcommand{\authorchinesename}{\textbf{\zh{邢骏}}} % small caps for name
\newcommand{\authorname}{\textbf{Jun Xing}} % small caps for name
\newcommand{\authorpage}{http://junxnui.github.io/}
\newcommand{\advisername}{Li-Yi Wei}
\newcommand{\adviserchinesename}{\zh{魏立一}}
\newcommand{\thesistitle}{Autocomplete Hand-drawn Sketches and Animations} % title for normal uses
\newcommand{\thesistitleCover}{Autocomplete Hand-drawn Sketches and Animations} % title for cover page
\newcommand{\thesistitleAbstract}{Autocomplete Hand-drawn Sketches and Animations}   % title for abstract page
\newcommand{\mydate}{February, 2017}


%\title{Predictive Sketching Interface based on Workflows}
\title{\thesistitle{}}

% ----------------------------------------------------------------------
% The section below defines www links/email for author and institutions
% They will appear on the title page of the PDF and can be clicked
\ifpdf
  \author{\href{http://junxnui.github.io}{\authorname{} \hspace{2mm} \authorchinesename{}}}
%  \cityofbirth{born in XYZ} % uncomment this if your university requires this
%  % If city of birth is required, also uncomment 2 sections in PhDthesisPSnPDF
%  % Just search for the "city" and you'll find them.
  \collegeordept{\href{http://www.cs.hku.hk}{Department of Computer Science}}
  \university{\href{http://www.hku.hk}{The University of Hong Kong}}

  % The crest is a graphics file of the logo of your research institution.
  % Place it in ./0_frontmatter/figures and specify the width
  \crest{\includegraphics[width=4cm]{logo}}
  
% If you are not creating a PDF then use the following. The default is PDF.
\else
  \author{\authorchinesename{}}
%  \cityofbirth{born in XYZ}
  \collegeordept{Department of Computer Science}
  \university{University of Hong Kong}
  \crest{\includegraphics[width=4cm]{logo}}
\fi

\supervisor{Supervised by Dr. \href{http://www.liyiwei.org/}{\advisername{} \adviserchinesename{}}}

%\renewcommand{\submittedtext}{change the default text here if needed}
\degree{Doctor of Philosophy (PhD)}
\degreedate{\mydate{}}


% ----------------------------------------------------------------------
       
% turn of those nasty overfull and underfull hboxes
\hbadness=10000
\hfuzz=50pt


%: --------------------------------------------------------------
%:                  FRONT MATTER: dedications, abstract,..
% --------------------------------------------------------------

\begin{document}

%\language{english}

% sets line spacing
\renewcommand\baselinestretch{1.2}
\baselineskip=18pt plus1pt


%: ----------------------- generate cover page ------------------------

\maketitle  % command to print the title page with above variables


%: ----------------------- cover page back side ------------------------
% Your research institution may require reviewer names, etc.
% This cover back side is required by Dresden Med Fac; uncomment if needed.

\nothing{
\newpage
\vspace{10mm}
%1. Reviewer: Prof. Albert Einstein

\vspace{10mm}
%2. Reviewer: Dr. Strange

\vspace{10mm}
%3. Reviewer: 

\vspace{20mm}
%Day of the defense: November 17, 2016

\vspace{20mm}
\hspace{70mm}%Signature from head of PhD committee:
}

\newpage
\thispagestyle{empty}
\mbox{}

%\backmatter


%: ----------------------- abstract ------------------------

% Your institution may have specific regulations if you need an abstract and where it is to be placed in the document. The default here is just after title.

%: Declaration of originality

% Thesis statement of originality -------------------------------------

% Depending on the regulations of your faculty you may need a declaration like the one below. This specific one is from the medical faculty of the university of Dresden.

\begin{declaration}        %this creates the heading for the declaration page


I declare that this thesis represents my own work, except where due acknowledgement is made, and that it has not been previously included in a thesis, dissertation or report submitted to this University or to any other institution for a degree, diploma or other qualifications. 
 
\vfill

\noindent Signed:\\
\rule[1em]{25em}{0.5pt}  % This prints a line for the signature

\authorname \\
\mydate
 
\end{declaration}


% ----------------------------------------------------------------------
\newpage
\thispagestyle{empty}
\mbox{}

\begin{abstract}
Drafting the abstract when you have some rough project idea.
It does not need to be fancy, just enough so that you can understand your own writing later.
The abstract should contain the basic information outlined in the introduction part below.

\liyi{(December 18, 2016)
I am starting to double this as a tutorial on how to use paper drafts to manage project progress.
This is still work in progress and not yet done, but I hope to finish a draft within the next few months.
}%liyi
\end{abstract}

\clearpage
% The original template provides and abstractseparate environment, if your institution requires them to be separate. I think it's easier to print the abstract from the complete thesis by restricting printing to the relevant page.
% \begin{abstractseparate}
%   \begin{abstract}
Drafting the abstract when you have some rough project idea.
It does not need to be fancy, just enough so that you can understand your own writing later.
The abstract should contain the basic information outlined in the introduction part below.

\liyi{(December 18, 2016)
I am starting to double this as a tutorial on how to use paper drafts to manage project progress.
This is still work in progress and not yet done, but I hope to finish a draft within the next few months.
}%liyi
\end{abstract}

% \end{abstractseparate}


%: ----------------------- tie in front matter ------------------------

%\include{matters/dedication}
% Thesis Acknowledgements ------------------------------------------------


%\begin{acknowledgementslong} %uncommenting this line, gives a different acknowledgements heading
\begin{acknowledgements}      %this creates the heading for the acknowlegments

First and foremost, I would like to express my sincere thanks to my PhD advisor Dr. Li-Yi Wei for his patient advice and continuous support for my research, internship and career. 
I appreciate all his time, ideas and encouragement to make my PhD experience productive and stimulating, and all the freedom he gave me that makes my research life fun and interesting. 
His pursuit of top works and perfect details was contagious and motivational for me, which is the priceless thing I learned from him.

XXX

\end{acknowledgements}
%\end{acknowledgmentslong}

% ------------------------------------------------------------------------




%: ----------------------- contents ------------------------

\setcounter{secnumdepth}{3} % organisational level that receives a numbers
\setcounter{tocdepth}{3}    % print table of contents for level 3
\tableofcontents            % print the table of contents
% levels are: 0 - chapter, 1 - section, 2 - subsection, 3 - subsection


%: ----------------------- list of figures/tables ------------------------

\listoffigures	% print list of figures

%\listoftables  % print list of tables


%: ----------------------- glossary ------------------------

% Tie in external source file for definitions: /0_frontmatter/glossary.tex
% Glossary entries can also be defined in the main text. See glossary.tex
%\include{matters/glossary} 

%\begin{multicols}{2} % \begin{multicols}{#columns}[header text][space]
%\begin{footnotesize} % scriptsize(7) < footnotesize(8) < small (9) < normal (10)

%\printnomenclature[1.5cm] % [] = distance between entry and description
%\label{nom} % target name for links to glossary

%\end{footnotesize}
%\end{multicols}



%: --------------------------------------------------------------
%:                  MAIN DOCUMENT SECTION
% --------------------------------------------------------------

% the main text starts here with the introduction, 1st chapter,...
\mainmatter

\renewcommand{\chaptername}{} % uncomment to print only "1" not "Chapter 1"


%: ----------------------- subdocuments ------------------------

% Parts of the thesis are included below. Rename the files as required.
% But take care that the paths match. You can also change the order of appearance by moving the include commands.

\section{Introduction}

\note{
What problem we are trying to solve.
Why it is important, and why people should care.
}%note

Writing research papers (not just conference/journal papers but also technical white papers, patent drafts, course reports, grant proposals, etc.) is a core activity for many poor souls including professors, researchers, engineers, students, etc.
Since we must do it one way or another, we might as well do it as happily and effectively as possible.
Here are my personal suggestions.

\note{
What prior works have done, and why they are not adequate.
(Note: this is just high level big ideas. Details should go to a previous work section.)
}%note

Some people, including very successful ones, write papers only at the end of a project, like 3 days prior to the deadline.
This almost always lead to total chaos and breakdown, unless you have other means to keep track and organize all stuff.

\note{
What our method has to offer, sales pitch for concrete benefits, not technical details.
Imagine we are doing a TV advertisement here.
}%note

I learned from my PhD adviser to start writing from day one, so that I can collect everything I have in one place.
These paper drafts are external RAMs and communication mediums for the collective brains of my teams.

\note{
Our main idea, giving people a take home message and (if possible) see how clever we are.
}%note

\note{
Our algorithms and methods to show technical contributions and that our solutions are not trivial.
}%note

Writing paper should be like writing programs.
Use Latex and revision control (e.g. bitbucket) your sources.

\note{
Results, applications, and extra benefits.
}%note

\begin{figure}[htb]
  \centering
  \subfloat[raster]{
    \label{fig:example:raster}
    \includegraphics[width=0.48\linewidth]{161.jpg}
  }%subfloat
  \subfloat[vector]{
    \label{fig:example:svg}
    \includegraphics[width=0.48\linewidth]{figs/handdrawn/example.pdf}
  }%subfloat

 \Caption{Example figure.}
 {%
\subref{fig:example:raster} is a raster image and \subref{fig:example:svg} is a vector graphics.
Never, ever, rasterize vector graphics unless you want large size and low quality files.
 }
 \label{fig:example}
\end{figure}


See \cite{Sun:2016:MVP} and \Cref{fig:example}.

	% background information
\chapter{Content}
\label{cha:content}

This part usually contains multiple chapters (and thus files) from your papers \cite{Xing:2014:APR,Xing:2015:AHA}.


\section{Limitations and Future Work}
\label{sec:conclusion}

Disclose all limitations, and describe how potential future works can address these and lead to more interesting and ground breaking stuff.

\clearpage
% --------------------------------------------------------------
%:                  BACK MATTER: appendices, refs,..
% --------------------------------------------------------------

% the back matter: appendix and references close the thesis


%: ----------------------- bibliography ------------------------

% The section below defines how references are listed and formatted
% The default below is 2 columns, small font, complete author names.
% Entries are also linked back to the page number in the text and to external URL if provided in the BibTex file.

% PhDbiblio-url2 = names small caps, title bold & hyperlinked, link to page 
\begin{multicols}{2} % \begin{multicols}{ # columns}[ header text][ space]
\begin{small} % tiny(5) < scriptsize(7) < footnotesize(8) < small (9)

\bibliographystyle{Latex/Classes/PhDbiblio-url2} % Title is link if provided
\renewcommand{\bibname}{References} % changes the header; default: Bibliography

\bibliography{paper} % adjust this to fit your BibTex file

\end{small}
\end{multicols}

% --------------------------------------------------------------
% Various bibliography styles exit. Replace above style as desired.

% in-text refs: (1) (1; 2)
% ref list: alphabetical; author(s) in small caps; initials last name; page(s)
%\bibliographystyle{Latex/Classes/PhDbiblio-case} % title forced lower case
%\bibliographystyle{Latex/Classes/PhDbiblio-bold} % title as in bibtex but bold
%\bibliographystyle{Latex/Classes/PhDbiblio-url} % bold + www link if provided

%\bibliographystyle{Latex/Classes/jmb} % calls style file jmb.bst
% in-text refs: author (year) without brackets
% ref list: alphabetical; author(s) in normal font; last name, initials; page(s)

%\bibliographystyle{plainnat} % calls style file plainnat.bst
% in-text refs: author (year) without brackets
% (this works with package natbib)


% --------------------------------------------------------------

% according to Dresden med fac summary has to be at the end
%\begin{abstract}
Drafting the abstract when you have some rough project idea.
It does not need to be fancy, just enough so that you can understand your own writing later.
The abstract should contain the basic information outlined in the introduction part below.

\liyi{(December 18, 2016)
I am starting to double this as a tutorial on how to use paper drafts to manage project progress.
This is still work in progress and not yet done, but I hope to finish a draft within the next few months.
}%liyi
\end{abstract}


\ifthenelse{\equal{\final}{0}}
{
\cleardoublepage
\pagenumbering{roman}

\chapter{Blog}
\label{sec:blog}

\begin{description}
\item[June 2, 2017]
{
\jun{
Hi from Jun.
}%jun

\mengqi{
Hi from Mengqi.
}%mengqi

\liyi{
Hi from Li-Yi.
}%liyi
}

\end{description}

}
{}

\end{document}
