\section{Introduction}

\note{
What problem we are trying to solve.
Why it is important, and why people should care.
}%note

Writing research papers (not just conference/journal papers but also technical white papers, patent drafts, course reports, grant proposals, etc.) is a core activity for many poor souls including professors, researchers, engineers, students, etc.
Since we must do it one way or another, we might as well do it as happily and effectively as possible.
Here are my personal suggestions.

\note{
What prior works have done, and why they are not adequate.
(Note: this is just high level big ideas. Details should go to a previous work section.)
}%note

Some people, including very successful ones, write papers only at the end of a project, like 3 days prior to the deadline.
This almost always lead to total chaos and breakdown, unless you have other means to keep track and organize all stuff.

\note{
What our method has to offer, sales pitch for concrete benefits, not technical details.
Imagine we are doing a TV advertisement here.
}%note

I learned from my PhD adviser to start writing from day one, so that I can collect everything I have in one place.
These paper drafts are external RAMs and communication mediums for the collective brains of my teams.

\note{
Our main idea, giving people a take home message and (if possible) see how clever we are.
}%note

\note{
Our algorithms and methods to show technical contributions and that our solutions are not trivial.
}%note

Writing paper should be like writing programs.
Use Latex and revision control (e.g. bitbucket) your sources.

\note{
Results, applications, and extra benefits.
}%note

\begin{figure}[htb]
  \centering
  \subfloat[raster]{
    \label{fig:example:raster}
    \includegraphics[width=0.48\linewidth]{161.jpg}
  }%subfloat
  \subfloat[vector]{
    \label{fig:example:svg}
    \includegraphics[width=0.48\linewidth]{figs/handdrawn/example.pdf}
  }%subfloat

 \Caption{Example figure.}
 {%
\subref{fig:example:raster} is a raster image and \subref{fig:example:svg} is a vector graphics.
Never, ever, rasterize vector graphics unless you want large size and low quality files.
 }
 \label{fig:example}
\end{figure}


See \cite{Sun:2016:MVP} and \Cref{fig:example}.

