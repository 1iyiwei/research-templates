\section{Method}
\label{sec:method}

\begin{align}
\energy = \mass \lightspeed^2
\label{eqn:emc2}
\end{align}

Always use defined instead of naked math symbols for clarify and consistency.
For example, in \Cref{eqn:emc2} I define all symbols inside \filename{symbols.tex} so later if I need to change $\energy$ from $E$ to $e$ I just need to change one line in \filename{symbols.tex} instead of chasing $E$ everywhere.
This can save you a lot of time and sanity if you have many math symbols.

\begin{algorithm}
  \Caption{Dart throwing algorithm.}{}
  \label{alg:dart_throw}
  \begin{algorithmic}[1]
    \REQUIRE sample domain $\domainsym$ with distance $\dist$ and conflict $\conflictdist$ measure
    \ENSURE output sample set $\sampleset$
    \STATE $fail \leftarrow 0$
    \WHILE{$\domainsym$ is not fully covered and $fail$ not too high}
    \STATE $\sample \leftarrow \funct{RandomSample}(\domainsym)$
    \IF{$\dist(\sample, \sampleprime) < \conflictdist \; \forall \sampleprime \in \sampleset$}
    \STATE $\sampleset \leftarrow \sampleset \union \sample$ \pcomment{accept trial sample}
    \ELSE
    \STATE $fail = fail + 1$
    \ENDIF
    \ENDWHILE
    \RETURN{} $\sampleset$
  \end{algorithmic}
\end{algorithm}




Pseudo-code can help summarize and explain the algorithms, but only if presented with sufficient clarity and simplicity. 
\Cref{alg:dart_throw} is an example.
All algorithms should be understandable from the main texts without looking at the pseudo-codes, which are usually more suitable for summarization than explanation.

